\chapter{Introduction}

\section{Microbes}
Microbes are defined as whatever is not visible at the human eye: bacteria cells' dimensions are in the order of micrometre, while viruses in the order of nanometre.
It is obvious how there is no visible part by eye.
This is particularly true for viruses: their dimension make them almost impossible to perceive by any other method than genomics.

	\subsection{Prevalence of microbes}
	We are living in a microbial world: more than $90\%$ of the biomass is composed of them and they are responsible for a great part of the biochemical cycle.
	Microbes can thrive in a variety of environment and according to some estimates they compose $10^{17}g$ of biomass.
	To put that in context the overall weight of the human species is three or four order of magnitude less.
	They also form the human microbiome, with important medical implication.

	\subsection{Difficulties in studying them}
	Of the predicted $30$ million species to exist only thousands can be cultured in isolation in the lab.
	There is a need to create a way to directly study and characterized samples taken from the environment.

\section{Genomics}
Once the genetic material is isolated and sequenced a huge amount of information needs to be interpreted.
The first sequenced gene was one of a bacteriophage.
Also the first complete genome wase one of a bacteriophage and is used by ILLUMINA as a control.
The first bacterial genome was published in $1995$ and was that of Haemophilus influenzae.
It has a dimension of $1.8Mb$ and sequencing took a year to complete.
The first archea was sequenced in $1996$.
In $1996$ the genome of S. cerevisiae has been sequenced and it was noticed that the genome shows a considerable amount of genetic redundancy.
The next step was to elucidate the biological functions of these genes.

	\subsection{Comparative genomics}
	Studying two different strains of the same organism difference in the genome can be linked to the difference in phenotype.

	\subsection{Metagenomics}
	Metagenomics is the study of the DNA from all the genomes in an environment.
	By sampling all of the DNA from a given environment, it is possible to study the presence of bacterial ecosystems, independent of the ability to culture bacterial in the lab.
	Large evolutionary radiation of bacterial lineages whose members are mostly uncultivated and only known through metagenomics and single cell sequencing have been described as nanobacterial.
	They have small genomes and lack several biosynthetic pathways and ribosomal proteins.

\section{Leveraging computational power}
Despite the advantages in DNA sequencing technology the sequencing of genomes has not progressed beyond clones on the order of the size of $\lambda$ because of the lack of sufficient computational approaches that would enable the efficient assembly of a large number of independent, random sequences into a single assembly.
When moving from low-throughput to high-throughput biology statistical power is needed: the genome of a bacterium must comprise all the DNA coding molecules present in the cell/
With millions of reads from NGS of an environmental sample, it is possible to get a complete overview of any complex micro biome with thousands of species.

	\subsection{Comparing low-throughput and high-throughput pipeline}
	In a low throughput pipeline to find the pathogenic agent for a novel outbreak a panel of reasonable putative causative agents is identified.
	Then one-by-one cultivation protocols to grow the agents from the infected tissue are performed.
	This is very time consuming.
	High-throughput instead sequences the full DNA repertoire of the sample and try to identify the pathogen by its genomic signature.
